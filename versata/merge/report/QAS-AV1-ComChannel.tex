\subsubsection{Av1: Communication channel between the remote module and the ReMeS system}
Because of a failure in the intermediate telecom infrastructure, or because
of a bad signal from the remote module to the intermediate telecom infrastructure,
key functionalities of the ReMeS system are compromised: reading data from a
remote monitoring module cannot be sent to ReMeS and ReMeS cannot control
the remote valves in case of a leak.
\begin{itemize}
	\item \textbf{Source:} external
	\item \textbf{Stimulus:} The external communication channel between the remote
								module and the ReMeS system is failing. This results in
								the inability of sending data from ReMeS to a remote 
								valve or from a remote monitoring module to ReMeS. It is also
								possible that there is a bad signal between the remote module
								and the external communication channel of the intermediate
								telecom infrastructure.
	\item \textbf{Artifact:} Remote modules, external communication channel of the intermediate telecom infrastructure.
	\item \textbf{Environment:} normal execution
	\item \textbf{Response:} 
		\begin{itemize}
			\item Prevention:
				\begin{itemize}
					\item The ReMeS company has negotiated a Service-Level Agreement (SLA) with the intermediate telecom operator that maintains the 
							external communication channel. This SLA states that the external communication channel has 98\% availability. 
					\item Before installing the remote module, the connection to the external communication
							channel should be checked if it's sufficient. Otherwise another solution needs to be found to improve this connection such
							as trying another type of connection (gprs, wifi, SMS).
				\end{itemize}
			\item Detection:
				\begin{itemize}
					\item ReMeS detects that a remote module hasn't sent any data for a certain time.
					\item ReMeS detects that all remote modules using a certain type of communication (gprs, wifi, 3G) haven't sent any data for a certain
							time. ReMeS can conclude that the external communication channel is failing and not the signal between the remote module and 
							the external communication channel.
					\item ReMeS keeps track of how long there has been a lack of communication.
				\end{itemize}
			\item Resolution:
				\begin{itemize}
					\item In case that one remote module hasn't sent any data for a certain time, the client owning the remote module is notified. Also a
							ReMeS operator is notified. If the problem isn't solved in time, the ReMeS operator will contact the client. If the problem still isn't
							resolved after the help of the ReMeS operator, a ReMeS technician will be sent to solve the problem with the remote module.
					\item In case that the external communication channel is failing, the ReMeS System Administrator is notified. The System Administrator will
							contact the telecom operator to resolve this problem.
				\end{itemize}
		\end{itemize}
	\item \textbf{Response measure:}
		\begin{itemize}
			\item Detection time
				\begin{itemize}
					\item The detection time of a single module equals to the transmission rate of the remote monitoring module plus 1 minute margin. As soon as a
							remote module should send a new data trame and ReMeS doesn't receive any, ReMeS can conclude that there is a problem with that connection.
					\item If many remote modules aren't sending data, ReMeS will detect this later as a failure in the external communication channel. Suppose T is the transmission frequency of a module,
							we will detect this problem within $2*T$ time.
				\end{itemize}					
			\item Notification time
				\begin{itemize}
					\item When only one remote module isn't sending any data, the owner of the module will be notified. Since this notification can happen through
							SMS or through email, there are different notification times. Through SMS the maximum time should be 5 minutes. If the owner is notified
							through email, the email will be delivered within 5 minutes, but we cannot predict when the owner shall read that notification email.
					\item If more remote modules aren't sending data, we can conclude that there is a problem with the external communication channel. A system administrator
							will be notified about this within 5 minutes. He/she will then contact the telecom company in order to resolve the issue.
				\end{itemize}
			\item Resolution time
				\begin{itemize}
					\item In the case that the external communication channel has a problem, we cannot ensure a fast recover of this channel. The SLA we have with the
							telecom company tells us that if there is a problem, it should be resolved within maximum 30 minutes.
					\item When there is a problem with the signal between the remote module and the external communication channel, the time to resolve this issue depends
							on the client, the ReMeS operator and the ReMeS technician. If the client cannot resolve the problem within 2 days, a ReMeS operator will call the client to
							help him/her. If the ReMeS operator and the client are unable to resolve the problem, a ReMeS technician will be dispatched. This technician will
							go on scene to resolve the issue. Since the ReMeS technician will arrive within 5 days. Therefor the maximum repair time is 7 days.
				\end{itemize}
		\end{itemize}
\end{itemize}