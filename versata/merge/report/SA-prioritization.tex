%hier typen...
\subsection{Prioritization of threats}
This section provides a list of the threats (ID + title) of the previous section. The order is based on the threat's risk (likelihood * impact). The distinction was made between high, medium, and low risk and,  within each category, the threats are also ordered according to their risk.
Finally, an explanation will be given of why the threats were ordered in this particular way.

\subsubsection{High priority}
The threat that was given the highest priority is T02 - Information disclosure of customer usage history. The main reason for this choice is the fact 
that is one of the more significant threats to the privacy of the customers. This threat is also very difficult to deal with. 
Other threats that were labeled as high priority threats include the threats about spoofing an internal user of ReMeS. Motivated attackers
wishing to exploit this threat could cause ReMeS harm in name and function. This should, evidently, be avoided at all cost.


%T02 high priority want privacy en niet simpel vermeden
%de rest high priority want spoofing users is vuil
%T02 boven T03 en T04, want privacy zeer belangrijk en T04, T03 moet toch al zeer skilled insider zijn met onze beveiliging (likelyhood)
\begin{itemize}
\item T02 - Information disclosure of customer usage history
\item T03 - Spoofing an internal user of ReMeS by falsifying credentials
\item T04 - Spoofing a user of ReMeS because of weak credential storage.
\end{itemize}


\subsubsection{Medium priority}
The threats that were labeled as being of medium priority are placed in this category because of the fact that some form of trust
has been put in the employees of ReMeS. Misuse of this trust will have serious consequences for the misactors and this should 
be incentive enough to mitigate the threat for these specific cases. Other than that, the medium priority threats could be mitigated
further or avoided altogether without too much trouble from the ReMeS side.
%bij T12 is juiste informatie belanrijk om rekeningen op te sturen, maar niet high priority, want kan simpel vermeden/opgelost worden
%de rest is belangrijk, want het gaat om privacy van customers, maar medium want we vertrouwen employees en/of simpel vermeden

\begin{itemize}
\item T11 - User unawareness
\item T09 - Missing user consents

\item T01 - Linking Alarm configuration data to user data
\item T10 - Non-compliance management
\item T12 - content inaccuracy
\end{itemize}

\subsubsection{Low priority}
Due to the fact that some level of trust is put in our employees, some threats are of a low priority. The list below contains
threats that are just rather unlikely to occur because of the nature of the consequences for the misactors or the fact
that the trouble that a misactor has to go through to exploit such a threat is not worth the merit of the exploit. (e.g.
side channel information disclosure of internal processes).


%Since we put some trust in our employees, we can put threats in low priority. Linkability of requests sent to the external UIS
%has an even lower priority since the chance of this threat happening is very very low. This because the attacker must know what
%bill is being made during the attack.

%low want we vertrouwen employees blabla
%bij T05 omdat de kans hierop enorm klein is

%T08 boven 07 en 06 want misactor van T08 is elke employee, terwijl 07 skilled insider is en 06 enkel authorized insider

\begin{itemize}
\item T08 - Non-compliance of employees
\item T07 - Side channel information disclosure internal process
\item T06 - Information disclosure internal process
\item T05 - Linkability of requests sent to external UIS
\end{itemize}
