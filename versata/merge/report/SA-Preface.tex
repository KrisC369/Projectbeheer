\section{Preface}
For the creation and documentation of this system, it was recommended to use Visual Paradigm 
as the design editor of choice. However due to some compatibility issues that arose
from trying to use Visual Paradigm as the editor between different members of the design team, 
Another software suite was used for creating the design diagrams.
We did however consult the didactical team about using another diagram design tool than 
Visual Paradigm. 
We also made sure that the software program we used adhered to the same uml specifications
as Visual Paradigm such that the diagram contents would be readable and understandable by anyone
who is used to Visual Paradigm. 

The design tool that was eventually used to document the development of the ReMeS system is called
UMLet, a free, open-source uml tool.
We took the time to confirm that the diagrams actually are similar to diagrams generated from Visual Paradigm. The only noticeable difference is that in Visual Paradigm, when drawing components, both the component symbol and stereotype (<<component>>) are shown. However in our diagrams, components are only represented by rectangles with the component-symbol in the top right corner. 
These are not to be confused with single blank rectangles which are used to indicate class elements.
Our specification is still conform to the uml specification about representing components, as is Visual Paradigm's.

