\section{Final Architecture Design}
\label{sec:FinalArch}
\subsection{Context diagram}
\inclimg{0.50}{0}{../SA-ContextDiagram}{The context diagram for ReMeS.}
The context diagram is used to describe the most important external factors
that interact with the system. In this case the most important system to view is the 
ReMeS system itself. This is the system that has been designing thus far.

The most important external actors are the remote device, the users and the UIS
(User Information System). The diagram is in fact quite self-explanatory as to how these
components use the border components in the ReMeS system in question. The interfaces
through which these forms of communication happen, are also included in the context diagram, 
albeit in name only. The detailed version of these interfaces can be found in section \ref{sec:InterfaceDescription}

\subsection{Overall component diagram}

The overall component diagram provides a good depiction of how the ReMeS is divided into 
components that together handle the functional and non-functional requirements.
\inclimg{0.45}{90}{L0ReMeS-Report-Iteration3}{The final architectural diagram for ReMeS.}

This section will explain how the core functionality is supported by the provided architecture.
The diagram shown in figure \ref{fig:L0ReMeS-Report-Iteration3} is a depiction of the architecture that came forth out of the previously discussed ADD runs (with a few minor modifications afterwards).

There are a couple of major components that play a vital role in this architecture. These components will be explained further in the following section about the decompositions.
One of the most important components shown in the diagram is the data storage component. 
Most of the components use this data storage component in one way or another. That is why the interface the data storage component provides, is a requirement for a lot of different components. This also shows in the diagram if one follows the lines going out from the IDatabase. 

 The Watchdog component that is present in the overall component diagram doesn't seem to interact with the rest of the components. The watchdog component however has interaction and communication functionality with every other component. All components for which failure needs to be detected, will send heartbeat updates to this central watchdog. We omitted drawing these interface dependencies because of the fact that the whole diagram would have become practically  unreadable. Keep in mind though that this is not some ghost component with no interaction with any component.

It is also important to note that an event channel is used throughout the system. This event channel is used for the dispatching of different types of messages such as failure indications and component state updates. This component is actually not shown in the overall component diagram because it interacts with a lot of different components but on a lower level than the overall component diagram is meant to show.
This event channel will be shown and explained in both the further decompositions in section \ref{sec:DecompCompDiag} and the element catalog in section \ref{sec:ElementCatalog}
\subsubsection{Core functionality}
The core functionality of the system can be described as the ability to handle measurement information utility networks in the form of trames sent by remote devices. Also the control of remote devices installed on valves, is an important core functionality. Also the management of and reporting of this data by and to the user, is an important aspect of the functionality this ReMeS system should offer.

The proposed architecture uses an incoming and an outgoing gateway to handle functionality of receiving and sending the data trames from and to the remote devices. The incoming gateway is able to route different types of trames to different schedulers. These schedulers decide when, where and how these trames should be processed. Eventually the information is extracted from the trames and stored in a data storage component.   

The different schedulers and processors also have the ability to schedule the notification of users (under certain circumstances such as leaks or failures) through a user notification module.
The next section will handle the different components in a little bit more detail. 

After the processing and storage of an incoming trame it may be the case that control trames should be sent automatically to effect certain effect in the remote devices. The components responsible for the processing of the incoming trames have the option of scheduling these outgoing trames when necessary. 
These outgoing trames will be sent through the outgoing gateway. 

The former describes the main data flow for trames throughout the system. Another important aspect is the user interaction. A user interaction component handles all user input and commands to the system. This component provides views of the ReMeS state based on the type of user interacting with the component and handles the interaction of this user with the system. When a user issues a command, it can interact with the data storage component, but also the effectuation of sending control trames will be handled by the system through a manner of dispatching the commands to the right component. 

A last important aspect is the use of an invoice manager which is in charge for the generation of invoices based on triggers by either the UI (for an operator) or certain internal events.

\subsection{Decomposition component diagrams}
\label{sec:DecompCompDiag}
This section will handle the decompositions of the major components in the overall component diagram. 
However a lot of decompositions are illustrated, discussed and explained in the ADD sections. 
To avoid overly repeating ourselves, the modules that have not changed since the last ADD iteration, will
only be referred to. 

The incoming gateway is shown in figure \ref{fig:L2IncomingGateway}. This is the gateway which receives the data trames from remote devices. This component also handles the acknowledgements of incoming trames and the callback for expected acknowledgement from sent out trames.

The scheduler for incoming measurement trames is shown in figure \ref{fig:L2IncomingMeasurementScheduler}. 
This component handles the incoming measurement trames and schedules them for processing. 
This component also handles the checking for missing measurements in the trame handler. 
The buffer in this components handles the storage of measurement data in case of a central database failure. 
This buffer is made redundant by using multiple buffering instances separated on a network. 
The scheduler component can use different modes of scheduling based on the operation mode and workload that is being advertised on the event channel. 

The data storage component is shown in figure \ref{fig:L2DataStorage}. 
This component handles the central data storage and effective processing of measurements. 
The DB request handler handles the requests for access to the database and also incorporates a caching mechanism for availability and performance. 
This component internally uses different physical databases for different types and natures of data.
Anomaly detection is also provided by this module. These physical database are internally shielded by a
database access layer for object oriented interaction with the databases.

The watchdog component is shown in figure \ref{fig:L2Watchdog}. This component provides the availability monitoring service for the system. This module uses the event channel to post messages about the uptime status and 
possible failures to all interested (or subscribed) parties. 
Each component that wants or needs its status monitored sends repeated measurements to the watchdog to notify this component of it's activity (in some circles described as petting the watchdog).
A missing heartbeat will be flagged by the active loop and in that manner the failure of a component can be detected.

The user interaction component is shown in figure \ref{fig:L2UserInterface}.
The user interaction component is the component that provides user interaction for this system. It is realised by using a MVC (model-view-controller) structure. Each user category has its own type of view on the system. These views will interact with the controller layer to perform actions and get information from the core system. The application can request data by directly querying the database component or by monitoring the event channel of the system.

The outgoing gateway is the only component that has really changed since our last iteration of ADD.
During the ADD process, the different manners of sending data was made abstraction of. Eventually,
for the final architecture discussion, it was decided to elaborate on the different subcomponents that
are used to send trames over different technology networks (such as sms, grps or ethernet).
The final structure of the outgoing gateway is shown in the following figure.
\inclimg{0.35}{0}{L2OutgoingGatewayBetter}{The component diagram of the final version of the outgoing gateway component.}

The rest of the components are of a more trivial nature and the explanation and discussion can be found directly in the ADD section.

\subsection{Deployment diagram}
\inclimg{0.35}{0}{../SA-DeploymentDiagram}{The final version of the ReMeS system deployment diagram.}
	
The deployment diagram shows how different architectural components are to be deployed
in a real world scenario. There are 4 major devices that make up the ReMeS system, not counting the 
application server and web server while two major communication protocols are used internally.
\subsubsection{Device Nodes}
These devices are the Gateway which forms the physical link to the outside world when speaking about trames.
This device handles all communication to and from the remote modules, be it over an IP channel, a gprs channel
or an sms channel. 

The next component in line is the inbound scheduling server. This device handles the scheduling of trames coming in from the gateway
or the trames pending to be sent out to remote devices. 
Also the scheduling for processing the measurement trames and the scheduling of consumption prediction
requests gets handled by this device.

This device also has a connection to three standalone buffering devices. 
These buffers are very important for redundancy and availability purposes.

The next component is the consumption prediction server. This device handles the generation of consumption predictions
based on the user history and measurement data. 

The next component in line is the Database server. This server is an important device in the ReMeS system because this
device will hold just about all the data albeit stored in different databases. Request to the database are handled through
the request handler component which communicates with the database instances themselves using the jdbc protocol.
This device is also the device with the most coupling to other components in the system. This is quite normal because the whole system
is built for the core functionality of storing, transforming and using this data in one way or another.

The next component is the Computation and fault detection unit. This device is actually a device which hosts and supports multiple execution environments.
In reality this could be realised by hosting multiple virtual machines layered on top of the host operation system running on this device.
The first execution environment has as responsibility the detection of anomalies in the incoming user data and the notification
of users when such anomalies are detected. The second execution environment provides the watchdog service that is being used to monitor 
the uptime of different components in the ReMeS system. Heartbeats are sent to the watchdog service by different components to inform it 
that they are still up and running. The heartbeats are sent using the icmp protocol.
The third execution environment provides the computation of consumption predictions. The fact that consumption prediction component resides inside a virtual execution environment allows the replication of this 
component for load balancing purposes. If more consumption prediction modules are needed, it just suffices to replicate the virtual machine on different physical computation nodes that are capable of running virtual machines. This improves the scalability and load distribution aspects for the computation of consumption prediction.
The fourth and final execution environment hosts the invoice manager. This component is capable of generating and dispatching invoices for customers and clients alike. 

Beside these major devices, also an application server and a web server will be deployed in order to provide the human users and the User Information System
with an easy to use interface to use the functionality provided by the ReMeS system.
The web server will use the SOAP protocol to communicate with the application server and outside users can visit the web pages provided
by this web server by using the http protocol.

\subsubsection{Communication protocols}
For the main method of communication between the different devices, the rmi protocol will be used. This form of communication will be made
possible by connecting the different devices to a switched network based on the IP protocol. On top of this switched network, the rmi protocol
will provide a means of communication for the internal components of the devices. Besides rmi, the icmp protocol will also be used by a lot of devices, 
to conform to the watchdog service specifications. The icmp protocol will be used to send heartbeats to the watchdog service for uptime monitoring purposes.

Furthermore two protocols for accessing the service from outside the ReMeS system will be used. The soap and the http protocol are both able to be transfered
on the previously discussed switched network. 

Finally, the remote devices have different options for communicating with the ReMeS system. They can use a simple udp, but also die gprs and sms protocol
can be used to send and receive trames to and from the ReMeS system.  

