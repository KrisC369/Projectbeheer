\subsubsection{Trame Handler}
\paragraph{Architectural drivers}

\begin{itemize}
	\item \textbf{UCw: Retrieve module checking schedule}\\
	For the scheduler for incoming measurement trames to check whether or not some
		measurements are missing, it must be able to retrieve the rate that the 
		remote module will be sending measurements. It can then compare the timestamps
		of the latest measurement trames with this rate to see if some measurements
		are missing.
	\item \textbf{AV2: Missing measurements}\\
	ReMeS should be able to detect a single missing measurement by monitoring the measurement schedule
	Operator should be notified after missing measurements.	
\end{itemize}
\paragraph{Tactics}
\begin{itemize}
	\item Availability 
	\begin{itemize}
		\item Fault detection \\
		An implementation of a Heartbeat system is used to check whether all devices send their measurements in time.
	\end{itemize}
\end{itemize}
\paragraph{Architectural Patterns}
\begin{itemize}
	\item{Heartbeat}\\
	The device's measurement trames are used as heartbeats in this implementation.
	An active loop component keeps track of the new entries in the heartbeat table and monitors them for absent measurements.
	The active loop component also has the capability of querying the data storage for the send schedule of the different devices.
	This component can also mark devices as active in the data storage component when a device sends it's first trame. 
	\item{Active Object}\\
	The active object pattern is used to provide a way of making an object run independently in its own thread of control for monitoring the heartbeat table for missing measurements.
	The active loop component represents an active object in this decomposition.
\end{itemize}
\inclimg{0.40}{0}{L3TrameHandler}{The Trame Handler after decomposition}

\paragraph{Verification and refinement of drivers}
All drivers for this module have been met and addressed. No further delegation to child nodes is needed.
