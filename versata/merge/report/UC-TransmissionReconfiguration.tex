\subsubsection{Transmission frequency reconfiguration}

\paragraph{Use Case Name}
Transmission frequency reconfiguration
\paragraph{Primary actor}
ReMeS client: The client of the ReMeS system ordering the reconfiguration. \\
ReMeS system: Handles the order.
\paragraph{Interested Parties}
Utility company: Can offer lower prices for consumers who update more frequently.\\
\paragraph{Preconditions}
The client has a remote monitoring module installed.
\paragraph{Normal Flow}
\begin{enumerate}
	\item The use case starts with the ReMeS client deciding to change the update frequency.
	\item include use case \textit{Log into ReMeS}.
	\item The ReMeS system show the user a list of installed modules.
	\item The client selects the module he wants to reconfigure.
	\item The client indicates that he is willing to send data more frequently.
	\item The ReMeS system processes this indication and sends a configuration command (trame) to the module in question.
	\item The previously chosen module receives a reconfiguration command correctly.
	\item The module processes the command and sets its update frequency accordingly.
	\item The module sends an acknowledgement back to the ReMeS system.
\end{enumerate}

\paragraph{Alternative Flow}
\begin{enumerate}
	\item[4a.] The selected module is not a valid module for reconfiguration. 
	\begin{enumerate}
		\item[4a1.] The user is notified. GOTO step 3.
	\end{enumerate}
	\item[7a.] The module does not receive the command correctly. 
	\begin{enumerate}
		\item[7a1.] The module does not do anything and does not send back an acknowledgement.
		\item[7a2.] After a fixed amount of time a system timer at the ReMeS system expires and the event is logged.
		\item[7a3.] The command is sent again. GOTO step 7.  
	\end{enumerate}
\end{enumerate}

\paragraph{Postcondition}
The update frequency of the module has been succesfully changed at the clients intent. 
The module will update more frequently from now on.
