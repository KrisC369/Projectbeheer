\subsubsection{Alarm data transmission}

\paragraph{Use Case Name}
Alarm data transmission for the gas monitoring module.
\paragraph{Primary actor}
Remote monitoring module (gas): Notices an anomaly and does the notifying.\\
ReMeS system: The system needs to notify interested parties and emergency services.
\paragraph{Interested Parties}
ReMeS Client: will be notified of the anomaly. \\
Emergency services (gas): Will be notified of gas anomalies.
\paragraph{Preconditions}
The module in question needs to be a gas remote control module.
\paragraph{Normal Flow}
\begin{enumerate}
	\item The use case starts when the module detects an anomaly 
	\item The module ignores all timers and pending tasks and immediately sends an alarm trame to the ReMeS system.
	\item The ReMeS system receives the alarm trame from the module.
	\item The ReMeS system looks up the contact information that is related to the module.
	\item The ReMeS system sends an alarm notification to the recipient, as configured in the remote module.
	\item include use case \textit{Notify Gas Emergency Services}
	\item Extension point: Use Case \textit{Gas Valve shutdown} if the client has a gas remote control module installed.
\end{enumerate}

\paragraph{Alternative Flow}
\begin{enumerate}
	\item[4a.] The module is configured not to send notifications to the client.  
	\begin{enumerate}
		\item[4a1.] Goto step 6.
	\end{enumerate}
\end{enumerate}

\paragraph{Postcondition}
The emergency services are notified and the alarm notification of the user is carried out if configured.
