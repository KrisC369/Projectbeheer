%voila... hier typen
\subsubsection{Assumptions}
This section discusses the different assumptions that were made. This includes general assumptions about the system (e.g. the data flow between X and Y is considered encrypted), and decisions and observations (e.g. Non repudiation is not considered a threat for this system). The reasoning behind each assumption and decision is also included.

These numbered assumptions will also be referenced to quite a lot in the threat elicitation section.
\begin{enumerate}
	\item \label{it:ass1} All internal processes can only be compromised by internal threats.
			We consider the back-end capable of protecting these internal
			processes from outsider threats. Therefor we will consider only
			one process, representing all internal processes, since the same
			threats apply to all of them.
	\item \label{it:ass2} Likewise to the previous assumption, all data flows between the
			internal processes are also considered as one data flow, since
			those threats apply to all of them. Like the previous assumption,
			data flows between internal processes are only vulnerable to internal
			threats.
	\item \label{it:ass3}Other dataflows (not between internal processes) are not grouped to
			one dataflow, since different threats may apply. Since these flows
			are not only between internal processes, the back-end cannot guarantee
			protection of these flows.
	\item \label{it:ass4}We assume that the data stores are encrypted and have a layer of access control
			installed. This leads us to trusting the data stores and as such, we will only
			consider one data store to represent all the stores. The data stores also have access control
			installed. This access control ensures that certain processes/entities only have access to the
			bare minimum amount of information needed for that process/entity. Queries requesting more
			information than a process or entity is allowed to have are blocked by this control.
			Since the data stores are encrypted
			very well and there is access control installed, identifiability and linkability are not an issue with our data stores
			(except for the administrator who has all access rights).
	\item \label{it:ass5}No non-repudiation threats exist in our system. Our system stores
			utility usages and creates invoices based upon these usages. This
			means our system components do not need plausible deniability.
	\item \label{it:ass6}Detectability also is not a pressing concern in our system. The privacy
			concerns of this system are all focused on the data itself, not on the
			detectability of it.
	\item \label{it:ass7}Non-compliance is an important threat in our system. However, this threat
			is not specific for a component of our system, but poses to the system
			as a whole. We will therefor not make a distinction between the different
			DFD elements for this threat.
	\item \label{it:ass8}The only dataflow between entities and internal processes that is susceptible to
			linkability and identifiability threats is the data flow between the external UIS service and
			the UIS portal process. Linking the company contacted with internal
			data flows might reveal personal information of a customer such as his
			utility provider. Since we allready assumed that the internal components are shielded off from
			the outside world and their threats, we can assume that linkability and identifiability are not an issue
			for the internal data flows of our system.
	\item \label{it:ass9}We assume that the data flows between external entities and internal processes
			are SSL-encrypted. We will also assume that trames sent via sms/gprs are
			also using an encrypted channel, provided by the service provider.
	\item \label{it:ass10}The only entities that directly add data to the system are the consumer and
			the remote device. Therefor we can assume that content unawareness is not a threat to the
			other entities.
	\item \label{it:ass11}The remote devices are preprogrammed to only send useful data. We can assume
			that content unawareness is not a threat for the remote device entity.
	\item \label{it:ass12} The identifiability of the entities \textit{consumer} is considered to be a threat. 
			The other entities all use a unique identifier and have no possible need to hide their identities. 
	\item \label{it:ass13} Information disclosure of data not meant to be accessed by certain internal users is not an issue because of the access control to the data stores, mentioned in assumption \ref{it:ass4}. 
Every user is able to access only specific views of the data store needed to execute their function.
	\item \label{it:ass14} We assume that the portals for employees are only accessible from inside the company network and not via the internet. 
This limits the threats to unauthorized access to customer or researcher functions from outside the system.
	\item \label{it:ass15} The authentication system is assume to be well implemented and secure.
	\item \label{it:ass16} The authentication credentials are also assumed to be stored securely on the server side.
	\item \label{it:ass17} Internal processes are not susceptible to corruption as we assume processes are implemented
correctly and input is suficiently validated, and memory access is dealt with as well.
	\item \label{it:ass18} Side channel attacks on data flows are not considered as they are highly-unlikely to occur
because they take a lot of analysis and the extracted information is not in correspondence of
the effort.

\end{enumerate}
