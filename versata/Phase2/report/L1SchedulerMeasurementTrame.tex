%\subsection{Level 1}
\subsubsection{Scheduler for incoming measurement trames}
This section will explain our reasons behind the decomposition of 
the "Scheduler for incoming measurement trames" component.
\paragraph{Architectural drivers}
This component has to ensure that the incoming measurement trames are
handed in the correct order to the data storage component. It also must
temporary store the incoming measurement trames in case that the measurement
database fails. Lastly, this component detects missing measurements.
\begin{itemize}
	\item \textbf{UC8': Send measurement}
	If a trame from a remote device is the first device sent, the system will mark the device as active.	
	\item \textbf{AV1: Measurement database failure}
	The scheduler for incoming measurement trames contains a buffer that can
		store at least 3 hours of data in case of a measurement database failure.
	\item \textbf{AV2: Missing measurements}
	ReMeS must be able to detect missing measurement updates from remote modules and notify an operator.	
	\item \textbf{P2': Anomaly detection}
	In normal modes ReMeS processes the incoming measurements in a first-in, first-out order.
	In overload modus, ReMeS gives priority based on the SLA with the customer.
	\item \textbf{UCy: Know the modus}
	For the system to know what order to hand the measurement trames to the
		data storage component, it must know the current modus of the system.
\end{itemize}
\paragraph{Tactics}
\begin{itemize}
	\item{Performance}
	\begin{itemize}
		\item{Scheduling Policy} \\
		To ensure the correct order or the incoming measurement trames,
			keeping in mind the modus of the system and the subscription
			type of the customers we chose for a scheduling policy
	\end{itemize}
\end{itemize}
\paragraph{Architectural Patterns}
\begin{itemize}
	\item Active Object\\
	The active object pattern allows us to insert new measurement trames
		in a scheduler. The scheduler will then decide itself which trames should
		be processed first. The active object will be implemented as a command and will be handled by a command processor structure in this case.
	The buffer that is used by the scheduler should also be large enough to hold at least 3 hours of data in case of a measurement database failure. This way, the availability driver will be addressed.
	\item Publish-Subscribe\\
	This pattern allows us to efficiently notify all interested parties
		about the modus of the system and more importantly, to be notified of changes in modus caused by other components. We will also use this pattern to
		notify an operator through the user interface by this publisher-subscriber
		pattern. The reason for notifying operators will be addressed in a further decomposition.
\end{itemize}
\inclimg{0.40}{0}{L2IncomingMeasurementScheduler}{The Scheduler for Incoming Measurement Trames after decomposition}

\paragraph{Verification and refinement of drivers}
After the decomposition, UCy and P2 are fully satisfied by the scheduler.
AV1 will be partially handled by the buffer of the scheduler component. This buffer
	will be large enough to contain 3 hours worth of data.
AV1 and AV2 will be (at least partially)  delegated to child nodes Trame Handler and Buffer for further decomposition. 
The Decomposition of Scheduler could show the implementation of a strategy pattern to effect the different policies of scheduling.

\textbf{Trame Handler}
\begin{itemize}
	\item \textbf{AV2: Missing measurements }
\end{itemize}

\textbf{Buffer}
\begin{itemize}
	\item \textbf{AV1: Measurement database failure }
\end{itemize}

