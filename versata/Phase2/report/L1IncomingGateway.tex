\subsubsection{Incoming gateway}
In this section, we will decompose the "Incoming Gateway" component.
\paragraph{Architectural drivers}
The main purpose of the incoming gateway is separating the measurement from
the alarm trames and other types of trames in the future, if need be. Therefore the chosen architectural drivers are the following:
\begin{itemize}
	\item \textbf{UCx: Know the type of the trame}\\
	This use case is a new use case. It is not described in the assignment.
		What this use case will do is decompose the incoming trame to find
		out what type of trame it is. 
	\item \textbf{UCz: Send acknowledgement}\\
	This use case is a new use case. If a measurement or alarm trame is received
		by ReMeS, ReMeS will send an acknowledgement of receiving the trame back
		to the sending device.
		
	\item \textbf{UC8': Send measurement}\\
	The incoming gateway will partially solve the use case "Send measurement"
		because it will make sure that the measurement trame is sent to the
		correct component of the system, after receiving the measurement trame and acknowledging it.
	\item \textbf{UC13': Send alarm}\\
	The incoming gateway will partially solve the use case "Send alarm"
		because it will make sure that the alarm trame is sent to the
		correct component of the system, after receiving the alarm trame and acknowledging it.
	\item \textbf{P1': Timely closure of valves}\\
	A mechanism should be in place to avoid starvation of (alarm) trame processing jobs.
	Alarm trames should be handled as soon as possible and ultimately within a hard deadline.
\end{itemize}
\paragraph{Tactics}
To efficiently solve these drivers, the following tactics were used.
\begin{itemize}
	\item{Performance}
	\begin{itemize}
		\item{Reduce computational overhead} \\
		In order to maximize the performance of the system, it is important to minimize the computational overhead caused by switching between functionality based on the type of incoming trame.

	\end{itemize}
\end{itemize}
\paragraph{Architectural Patterns}
The architectural patterns that are used to effect the tactics are the following:
\begin{itemize}
	\item Message Router\\
	The message router pattern localizes the router logic for the incoming trames.
	By using a message router implementation in the incoming gateway, it is possible to 
	cleanly separate different functionalities from each other.
	It is also has a high impact on modifiability aspects of the system. Although this is not one of the main QAS, it is considered an important factor in good quality architectures to be extensible and minimize the ripple effect of adding new changes. By using this pattern, adding new functionality based on new types of trames is made easier.  
	
	Message routers consume messages from one message channel (the channel where all
		external messages are received) and reinsert them into different message
		channels, depending on a set of conditions (alarm or measurement trames).
	\item Invoker \\
	To be able to easily receive messages from the remote modules, we opted
		for an invoker pattern.
\end{itemize}
\inclimg{0.40}{0}{L2IncomingGateway}{The Incoming Gateway after decomposition}


\paragraph{Verification and refinement of drivers}
After this decomposition, UCx (Know the type of trame) will be satisfied by the Message Router component. 
The partial functionalities and quality attributes described in UC8'(Send measurement), UC13'(Send alarm) and P1'(Timely closure of valves) are all handled and satisfied in the Message Router component.
The sending of acknowledgements on receiving certain trames, which is described in UCz(Send Acknowledgement) is handled by the component Acknowledgement Handler.

\textbf{Message Router}
\begin{itemize}
	\item \textbf{UCx: } Know the type of trame
	\item \textbf{UC8': } Send measurement 
	\item \textbf{UC13': } Send alarm 
	\item \textbf{P1': } Timely closure of valves
\end{itemize}

\textbf{Acknowledgement Handler}
\begin{itemize}
	\item \textbf{UCz: } Send Acknowledgement
\end{itemize}

