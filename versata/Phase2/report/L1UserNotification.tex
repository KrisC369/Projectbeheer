%\subsection{Level 1}
\subsubsection{User notification}
\paragraph{Architectural drivers}
This component will manage outgoing customer notifications. It wont prioritize
	any requests. The component handles the request First-in-First-out (FIFO).
The component will construct the messages and determine which communication channel
		to use.
\begin{itemize}
	\item \textbf{UC9': Notify customer}
	Whenever ReMeS wants to send a notification to a customer, ReMeS should
		address this component.
	\item \textbf{UC13': Send alarm}
	When ReMeS receives an alarm trame, the customer needs to be notified (if
		the customer has indicated this in his profile).
\end{itemize}


\paragraph{Tactics}
\begin{itemize}
	\item Performance
	\begin{itemize}
		\item Resource arbitration\\
		Different scheduling policies can be used. Currently the FIFO scheduling policy is used in this component.
	\end{itemize}
\end{itemize}


\paragraph{Architectural Patterns}
\begin{itemize}
	\item Message Translator \\
	To construct the notification to send to the remote device, we opted for
		a message translator pattern. This pattern allows us to create standard
		user notifications such as "Gas leak detected in your home."
	\item Active Object \\
	The active object pattern allows us to insert new user notification requests in a scheduler. The scheduler can
		then decide which requests to handle first.  Currently the scheduler will handle
		the requests FIFO.
	\item Proxy \\
	To increase the modifiability of the internal working of this component we chose
		to use the explicit interface pattern. The interface that we see outside the
		component will simply delegate it's methods to an internal component.
\end{itemize}
\inclimg{0.40}{0}{L2UserNotification}{The User Notification component after decomposition}


\paragraph{Verification and refinement of drivers}
Both UC9' and UC13' have been handled by this component. The second part of
UC9 (sending the message) will be handled by the outgoing gateway.
