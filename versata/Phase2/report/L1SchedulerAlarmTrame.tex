\subsubsection{Scheduler for incoming alarm trames}
This section will explain our reasons behind the decomposition of 
the "Scheduler for incoming alarm trames" component.
\paragraph{Architectural drivers}
The main purpose of this component is making sure that the trames are
processed in the correct order, according to their priority. Incoming
gas alarm trames must be treated with the highest priority and the
incoming water and power alarm trames with normal priority. This component will make sure
that this happens. This component will handle the following drivers:
\begin{itemize}
	\item \textbf{UC13': Send alarm}
	The scheduler for incoming alarm trames will partially help in solving
		this use case since it will notify the correct system component to
		send a control trame to the remote valve.
	\item \textbf{P1': Timely closure of valves}
	ReMeS will process incoming gas alarm trames with a higher priority as
		water and power alarm trames. ReMeS must also make sure that starvation of
		alarm trames is not possible.
\end{itemize}
\paragraph{Tactics}
\begin{itemize}
	 \item{Performance}
	 \begin{itemize}
		\item{Scheduling Policy} \\
		To ensure that certain incoming alarm trames such as gas alarm trames
			are treated with the highest priority, we chose for a scheduling
			policy. The scheduling policy will also make sure that starvation
			is not possible.
	 \end{itemize}
\end{itemize}


\paragraph{Architectural Patterns}
\begin{itemize}
	\item Active Object\\
	The active object pattern allows us, as in the Scheduler for incoming measurement trames,
		to insert new alarm trames in a scheduler. The scheduler will then decide
		itself which trames need to be processed first. The active object will be
		implemented in the same way as in the Scheduler for incoming measurement trames.
		The buffer that is used by the scheduler shouldn't be as big as with measurement trames.
	\item Proxy \\
	To increase the modifiability of the internal working of this component we chose
		to use the explicit interface pattern. The interface that we see outside the
		component will simply delegate it's methods to an internal component.
\end{itemize}
\inclimg{0.40}{0}{L2IncomingAlarmScheduler}{The Scheduler for Incoming Alarm Trames after decomposition}


\paragraph{Verification and refinement of drivers}
After the decomposition, P1 will be partially satisfied by the scheduler.
Some incoming alarm trames will be prioritized while starvation of certain
	incoming alarm trames is not possible.
UC13 will be delegated to another component in our system.