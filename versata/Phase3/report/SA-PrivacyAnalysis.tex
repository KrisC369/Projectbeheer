\section{Privacy analysis}
\subsection{Data Flow Diagram}

The data flow diagrams (DFD) depicted in figures \ref{fig:DFD-Level0} and \ref{fig:DFD-Level1} are based on the client-server view of the ReMeS system.

This section should contain the DFD diagram + an explanation of the decisions you made.
\inclimg{0.75}{0}{DFD-Level0}{The Level 0 DFD diagram for ReMeS.}
\inclimg{0.45}{0}{DFD-Level1}{The Level 1 DFD diagram for ReMeS.}

The DFD slightly varies from the original client-server view, as the following decisions were made: 
\begin{enumerate}
	\item All frontend components were decomposed into an external entity and the actual process. 
	\item The price rate notifier is not depicted in the client-server view, and is assumed to be part of the UIS web component.
	\item The UIS web service entity represents the UIS component that actually uses the ReMeS interface to communicate and not the offered web service to the UIS component.
	\item The authentication component is missing from the diagrams because it is assumed that the session token is sent from the users to the portal for each request.
	\item The Data processor component and the anomaly detector component were combined to one process: Data processor, 
because the additional process would not introduce any more information about or threats to the system.
	\item The \textit{Store query} represents an insert request to the database of some sorts, while the \textit{Read query}
represents a retrieval query for the database. 
If just \textit{Query} is used, it is assumed that the flow will contain both read and insert requests.
\end{enumerate}

\subsection{Mapping of threats to DFD}

This section gives an overview of the different threats that exist for each DFD element.
The DFD element names are represented in the DFD of the previous section!
This table is split up in to different sections for data stores, data flows, processes and entities.



% place the number of the misuse cases in the corresponding field(s).
% indicate the not examined threats with an "X"
% at the bottom: list the total number of uncovered threats per LINDDUN category
% + list the examined number of threats per LINDDUN category

%\begin{table}[h!]
%\begin{longtable}[h!]
%\begin{center}
%\caption{ \label{table:LINDDUN_mapping_MUCs}}
%\begin{tabular}{p{2.3cm} p{9cm} p{0.2cm} p{0.2cm} p{0.2cm} p{0.2cm}p{0.2cm} p{0.2cm} p{0.2cm} }
\begin{longtable}{p{2.3cm} p{9cm} p{0.2cm} p{0.2cm} p{0.2cm} p{0.2cm}p{0.2cm} p{0.2cm} p{0.2cm} }
%\caption{ \label{table:LINDDUN_mapping_MUCs}}
%\toprule[1pt]
\hline\hline
& Threat target & L & I & N & D & D & U & N \\
%\endhead
%\midrule[0.5pt]
\hline
Data Store &  Database (7.1) & $\times$ & $\times$ & $\times$ & $\times$ & $\times$ &  & $\times$ \\
	   &  Alarm configuration DB (7.14) & $\times$ & $\times$ & $\times$ & $\times$ & $\times$ &  & $\times$ \\
%\midrule[0.5pt]
\hline
Data Flow  &  Operator -- Operator portal (1 - 7.2) & $\times$ & $\times$ & $\times$& $\times$ &  $\times$ & & $\times$\\
		   &  Operator portal -- Operator (7.2 - 1) & $\times$ & $\times$ & $\times$& $\times$ &  $\times$ & & $\times$\\
		   &  Researcher -- Reasearcher portal (2 - 7.4) & $\times$ & $\times$ & $\times$& $\times$ &  $\times$ & & $\times$\\
		   &  Researcher portal -- Reasearcher (7.4 - 2) & $\times$ & $\times$ & $\times$& $\times$ &  $\times$ & & $\times$\\
		   &  Consumer -- Consumer portal (3 - 7.5) & $\times$ & $\times$ & $\times$& $\times$ &  $\times$ & & $\times$\\
		   &  Consumer portal -- Consumer (7.5 - 3) & $\times$ & $\times$ & $\times$& $\times$ &  $\times$ & & $\times$\\
		   &  Consumer -- Gateway (3 - 7.9) & $\times$ & $\times$ & $\times$& $\times$ &  $\times$ & & $\times$\\
		   &  Gateway -- Consumer (7.9 - 3) & $\times$ & $\times$ & $\times$& $\times$ &  $\times$ & & $\times$\\
		   &  Remote device -- Gateway (4 - 7.9) & $\times$ & $\times$ & $\times$& $\times$ &  $\times$ & & $\times$\\
		   &  Gateway -- Remote device (7.9 - 4) & $\times$ & $\times$ & $\times$& $\times$ &  $\times$ & & $\times$\\
		   &  UIS web service -- UIS portal (5 - 7.7) & $\times$ & $\times$ & $\times$& $\times$ &  $\times$ & & $\times$\\
		   &  UIS portal -- UIS web service (7.7 - 5) & $\times$ & $\times$ & $\times$& $\times$ &  $\times$ & & $\times$\\
		   &  3rd party billing service -- Billing portal (6 - 7.6) & $\times$ & $\times$ & $\times$& $\times$ &  $\times$ & & $\times$\\
		   & Billing portal --  3rd party billing service (7.6 - 6) & $\times$ & $\times$ & $\times$& $\times$ &  $\times$ & & $\times$\\
		   &  Operator portal  -- UIS Web service (7.2 - 7.7) & $\times$ & $\times$ & $\times$& $\times$ &  $\times$ & & $\times$\\
		   &  UIS Web service -- Operator portal (7.7 - 7.2) & $\times$ & $\times$ & $\times$& $\times$ &  $\times$ & & $\times$\\
		   &  Operator portal -- Database (7.2 - 7.1) & $\times$ & $\times$ & $\times$& $\times$ &  $\times$ & & $\times$\\
		   &  Database -- Operator portal (7.1 - 7.2) & $\times$ & $\times$ & $\times$& $\times$ &  $\times$ & & $\times$\\
		   &  Operator portal -- Statistics component (7.2 - 7.3) & $\times$ & $\times$ & $\times$& $\times$ &  $\times$ & & $\times$\\
		   &  Statistics component -- Operator portal (7.3 - 7.2) & $\times$ & $\times$ & $\times$& $\times$ &  $\times$ & & $\times$\\
		   &  Researcher portal -- Statistics component (7.4 - 7.3) & $\times$ & $\times$ & $\times$& $\times$ &  $\times$ & & $\times$\\
		   &  Statistics component -- Researcher portal (7.3 - 7.4) & $\times$ & $\times$ & $\times$& $\times$ &  $\times$ & & $\times$\\
		   &  UIS portal -- Statistics component (7.7 - 7.3) & $\times$ & $\times$ & $\times$& $\times$ &  $\times$ & & $\times$\\
		   &  Statistics component -- UIS portal (7.3 - 7.7) & $\times$ & $\times$ & $\times$& $\times$ &  $\times$ & & $\times$\\
		   &  UIS portal -- Database (7.7 - 7.1) & $\times$ & $\times$ & $\times$& $\times$ &  $\times$ & & $\times$\\
		   &  Database -- UIS portal (7.1 - 7.7) & $\times$ & $\times$ & $\times$& $\times$ &  $\times$ & & $\times$\\
		   
		   &  Consumer portal -- Statistics component (7.5 - 7.3) & $\times$ & $\times$ & $\times$& $\times$ &  $\times$ & & $\times$\\
		   &  Statistics component -- Consumer portal (7.3 - 7.5) & $\times$ & $\times$ & $\times$& $\times$ &  $\times$ & & $\times$\\
		   &  Consumer portal -- Database (7.5 - 7.1) & $\times$ & $\times$ & $\times$& $\times$ &  $\times$ & & $\times$\\
		   &  Database -- Consumer portal (7.1 - 7.5) & $\times$ & $\times$ & $\times$& $\times$ &  $\times$ & & $\times$\\
		   &  Statistics component -- Database (7.3 - 7.1) & $\times$ & $\times$ & $\times$& $\times$ &  $\times$ & & $\times$\\
		   &  Database -- Statistics component (7.1 - 7.3) & $\times$ & $\times$ & $\times$& $\times$ &  $\times$ & & $\times$\\
		  % &  Consumer -- Configuration controller (3 - 7.8) & $\times$ & $\times$ & $\times$& $\times$ &  $\times$ & & $\times$\\
		   &  Configuration controller -- Consumer portal (7.8 -7.5) & $\times$ & $\times$ & $\times$& $\times$ &  $\times$ & & $\times$\\
		   &  Consumer portal -- Configuration controller (7.5 -7.8) & $\times$ & $\times$ & $\times$& $\times$ &  $\times$ & & $\times$\\
		   &  Configuration controller -- Database (7.8 - 7.1) & $\times$ & $\times$ & $\times$& $\times$ &  $\times$ & & $\times$\\
		   &  Database -- Configuration controller (7.1 - 7.8) & $\times$ & $\times$ & $\times$& $\times$ &  $\times$ & & $\times$\\
		   &  Gateway -- Configuration controller (7.9 - 7.8) & $\times$ & $\times$ & $\times$& $\times$ &  $\times$ & & $\times$\\
		   &  Configuration controller -- Gateway (7.8 - 7.9) & $\times$ & $\times$ & $\times$& $\times$ &  $\times$ & & $\times$\\
		   &  Gateway -- Data processor (7.9 - 7.10) & $\times$ & $\times$ & $\times$& $\times$ &  $\times$ & & $\times$\\
		   &  Data processor -- Gateway (7.10 - 7.9) & $\times$ & $\times$ & $\times$& $\times$ &  $\times$ & & $\times$\\
		   &  Data processor -- Database (7.10 - 7.1) & $\times$ & $\times$ & $\times$& $\times$ &  $\times$ & & $\times$\\
		   &  Database -- Data processor (7.1 - 7.10) & $\times$ & $\times$ & $\times$& $\times$ &  $\times$ & & $\times$\\
		   &  Gateway -- Alarm processor (7.9 - 7.11) & $\times$ & $\times$ & $\times$& $\times$ &  $\times$ & & $\times$\\
		   &  Alarm processor -- Gateway (7.11 - 7.9) & $\times$ & $\times$ & $\times$& $\times$ &  $\times$ & & $\times$\\
		   &  Alarm processor -- Alarm Configuration Database (7.11 - 7.14) & $\times$ & $\times$ & $\times$& $\times$ &  $\times$ & & $\times$\\
		   &  Alarm Configuration Database -- Alarm processor (7.14 - 7.11) & $\times$ & $\times$ & $\times$& $\times$ &  $\times$ & & $\times$\\
		   &  Alarm processor -- Database (7.11 - 7.1) & $\times$ & $\times$ & $\times$& $\times$ &  $\times$ & & $\times$\\
		   &  Database -- Alarm processor (7.1 - 7.11) & $\times$ & $\times$ & $\times$& $\times$ &  $\times$ & & $\times$\\
		   &  Alarm processor -- Actuator controller (7.11 - 7.12) & $\times$ & $\times$ & $\times$& $\times$ &  $\times$ & & $\times$\\
		   &  Actuator controller -- Alarm processor (7.12 - 7.11) & $\times$ & $\times$ & $\times$& $\times$ &  $\times$ & & $\times$\\
		   &  Gateway -- Actuator controller (7.9 - 7.12) & $\times$ & $\times$ & $\times$& $\times$ &  $\times$ & & $\times$\\
		   &  Actuator controller -- Gateway (7.12 - 7.9) & $\times$ & $\times$ & $\times$& $\times$ &  $\times$ & & $\times$\\
		   &  Actuator controller -- Database (7.12 - 7.1) & $\times$ & $\times$ & $\times$& $\times$ &  $\times$ & & $\times$\\
		   &  Database -- Actuator controller (7.1 - 7.12) & $\times$ & $\times$ & $\times$& $\times$ &  $\times$ & & $\times$\\
		   &  Billing portal -- Database (7.6 - 7.1) & $\times$ & $\times$ & $\times$& $\times$ &  $\times$ & & $\times$\\
		   &  Database -- Billing portal (7.1 - 7.6) & $\times$ & $\times$ & $\times$& $\times$ &  $\times$ & & $\times$\\
		   &  Global demand predictor -- Database (7.13 - 7.1) & $\times$ & $\times$ & $\times$& $\times$ &  $\times$ & & $\times$\\
			&  Database -- Global demand predictor (7.1 - 7.13) & $\times$ & $\times$ & $\times$& $\times$ &  $\times$ & & $\times$\\		   
%\midrule[0.5pt]
\hline
Process & Operator portal (7.2) & $\times$ & $\times$ & $\times$& $\times$ &  $\times$ & & $\times$\\
        & Researcher portal (7.4) & $\times$ & $\times$ & $\times$& $\times$ &  $\times$ & & $\times$\\
        & Actuator controller (7.12) & $\times$ & $\times$ & $\times$& $\times$ &  $\times$ & & $\times$\\
        & Consumer portal (7.5) & $\times$ & $\times$ & $\times$& $\times$ &  $\times$ & & $\times$\\
        & Billing portal (7.6) & $\times$ & $\times$ & $\times$& $\times$ &  $\times$ & & $\times$\\
        & UIS portal (7.7) & $\times$ & $\times$ & $\times$& $\times$ &  $\times$ & & $\times$\\
        & Statistics component (7.3) & $\times$ & $\times$ & $\times$& $\times$ &  $\times$ & & $\times$\\
        & Alarm Processor (7.11) & $\times$ & $\times$ & $\times$& $\times$ &  $\times$ & & $\times$\\
        & Data processor (7.10) & $\times$ & $\times$ & $\times$& $\times$ &  $\times$ & & $\times$\\
        & Configuration controller (7.8) & $\times$ & $\times$ & $\times$& $\times$ &  $\times$ & & $\times$\\
        & Global demand predictor (7.13) & $\times$ & $\times$ & $\times$& $\times$ &  $\times$ & & $\times$\\
%\midrule[0.5pt]
Entity 	& Operator (1) & $\times$ & $\times$ & & &  & $\times$ & \\
	& Researcher (2) & $\times$ & $\times$ & & &  & $\times$ & \\
	& Consumer (3) & $\times$ & $\times$ & & &  & $\times$ & \\
	& Remote Module (4) & $\times$ & $\times$ & & &  & $\times$ & \\
	& UIS web service (5) & $\times$ & $\times$ & & &  & $\times$ & \\
	& 3rd party billing service (6) & $\times$ & $\times$ & & &  & $\times$ & \\
%\bottomrule[1pt]
\hline\hline
\caption{ \label{table:LINDDUN_mapping_MUCs}}
\end{longtable}
%\end{tabular}
%\end{center}

%\end{longtable}
%\end{table}



\newpage
After making different assumptions about the system and analysing the possible threats further, a lot of possible threats were ruled out.
The final threat table can be seen in the following figure. In this table only the actually handled threats are represented.

% place the number of the misuse cases in the corresponding field(s).
% indicate the not examined threats with an "X"
% at the bottom: list the total number of uncovered threats per LINDDUN category
% + list the examined number of threats per LINDDUN category

%\begin{table}[h!]
%\begin{center}
%\caption{ \label{table:LINDDUN_mapping_MUCs}}
%\begin{tabular}{p{2.3cm} p{9cm} p{0.2cm} p{0.2cm} p{0.2cm} p{0.2cm}p{0.2cm} p{0.2cm} p{0.2cm} }
\begin{longtable}{p{2.3cm} p{9cm} p{0.2cm} p{0.2cm} p{0.2cm} p{0.2cm}p{0.2cm} p{0.2cm} p{0.2cm} }
\hline\hline
%\toprule[1pt]
& Threat target & L & I & N & D & D & U & N \\
\hline
%\midrule[0.5pt]
Data Store &  General Database (7.1) & $\times$ & $\times$ & & & $\times$ & & $\times$ \\
\hline
%\midrule[0.5pt]
Data Flow  &  Operator -- Operator portal (1 - 7.2) & & & & & & & $\times$\\
		   &  Researcher -- Reasearcher portal (2 - 7.4) & & & & & & & $\times$\\
		   &  Consumer -- Consumer portal (3 - 7.5) & & & & & & & $\times$\\
		   &  Consumer -- Gateway (3 - 7.9) & & & & & & & $\times$\\
		   &  Remote device -- Gateway (4 - 7.9) & & & & & & & $\times$\\
		   &  UIS web service -- UIS portal (5 - 7.7) & $\times$ & $\times$ & & & & & $\times$\\
		   &  3rd party billing service -- Billing portal (6 - 7.6) & & & & & & & $\times$\\
		   &  General internal dataflow & & $\times$ & & & $\times$ & & $\times$\\
\hline
%\midrule[0.5pt]
Process & General internal process (7.2) & & & & & & & $\times$\\
\hline
%\midrule[0.5pt]
Entity 	& Operator (1) & &  & & &  & & \\
	& Researcher (2) & &  & & &  & & \\
	& Consumer (3) & & $\times$ & & &  & $\times$ & \\
	& Remote Module (4) & & $\times$ & & & & & \\
	& UIS web service (5) & & & & & & & \\
	& 3rd party billing service (6) & & & & & & & \\
\hline\hline
%\bottomrule[1pt]
%\end{tabular}
%\end{center}
\caption{ \label{table:LINDDUN_final_mapping_MUCs}}
\end{longtable}


