

\paragraph{T05 - Linkability of requests sent to external UIS}
\label{par:t06}
% the attributes noted with an asterix (*) are obligated
    \subparagraph{Summary:} The misactor links several requests to the same customer and creates a profile of this customer.
    \subparagraph{Primary mis-actor:} unskilled insider (external UIS) / skilled outsider
    \subparagraph{Basic path:}
    \begin{enumerate}
        \item[bf1.] A bill is being made for a certain customer. Requests are sent to the correct UIS (there exist
						different companies, requests need to be sent to the UIS of the company who has the customer).
        \item[bf2.] The misactor intercepts the data flow.
        \item[bf3.] The misactor can link several requests to the same customer.
    \end{enumerate}
    \subparagraph{Consequence:} The misactor can build a profile of the patient. The misactor knows the utility company of the customer.


    \subparagraph{Reference to threat tree node(s):} L\_df1, L\_df8
    \subparagraph{Parent threat tree(s):} L\_df
    \subparagraph{DFD element(s):} data flow from UIS portal to UIS web service (7.7 - 5).
    \subparagraph{Remarks:}
    \begin{enumerate}
        \item[r1.] The misactor doesn't have to know about the content being sent, only which UIS is being consulted.
        \item[r2.] To link the different requests, the misactor has to know for which customer a bill is being made, 
					which makes this threat very unlikely.
		\item[r3.] The right branch of the tree (insecure anonymity system (L\_df4)) and the other leaf nodes of
the non-anonymous communication branch (L\_df3) are not considered, as it is not the sender
(browse service) whose identity should be protected, but the patient, who is not directly part
of the data flow.
    \end{enumerate}



