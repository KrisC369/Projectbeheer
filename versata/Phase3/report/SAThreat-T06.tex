

\paragraph{T04 - Spoofing a user of ReMeS because of weak credential storage. } 
\label{par:t04}
% the attributes noted with an asterix (*) are obligated
    \subparagraph{Summary:} The misactor obtains user credentials from a client side system allowing him to log in and access the system.
    \subparagraph{Primary mis-actor:} Skilled insider or skilled outsider
    \subparagraph{Basic path:} 
    \begin{enumerate}
        \item[bf1.]{ The misactor gains access to the credentials of a user by weak credential storage at the client side 
(e.g. unprotected text file for easy recall of credentials)(S\_13). }
        \item[bf2.]{ The misactor uses the authentic credentials to log in to the system as an other user than himself.}
        \item[bf3.]{ The misactor receives all privileges of the spoofed employee or customer}
    \end{enumerate}
    \subparagraph{Consequence:} Confidential data is possibly exposed to outsiders.

    \subparagraph{Reference to threat tree node(s):} S\_8, S\_12, S\_13
    \subparagraph{Parent threat tree(s):} ID\_ds, S
    \subparagraph{DFD element(s):}1. Operator, 2. Researcher, 3.Consumer
    \subparagraph{Remarks:}
    \begin{enumerate}
        \item[r1.] An authentication system is present in the architecture, which rules out threat S\_4.
        \item[r2.] The authentication process is considered secure (assumption \ref{it:ass15}) thus the tampering threat (leaf of S\_3) does not hold, 
and it does not support null credentials (S\_10) or equivalence (S\_09), downgrade authentication (S\_11) 
or weak change management (S\_09). Also no key distribution storage is present (S\_14).
	\item[r3.] Spoofing due to falsifying crednetials is described in \nameref{par:t03}.
	\item[r4.] This form of spoofing can apply to outside users (consumers) or inside users (operators and researchers), but spoofing outside users does not affect the inside entities or vice versa.
    \end{enumerate}



