

\paragraph{T05 - Disclosure of the transmitted log-in credentials}
\label{par:t05}
TODO remove this thing
% the attributes noted with an asterix (*) are obligated
    \subparagraph{Summary:} The misactor gains access to the data flow that contains the credentials used for log-in.
    \subparagraph{Primary mis-actor:} Skilled outsider
    \subparagraph{Basic path:}
    \begin{enumerate}
        \item[bf1.] The misactor gains access to the data flow between the user and the authentication process.
        \item[bf2.] The misactor intercepts the credentials (username, password) of the user.
    \end{enumerate}
    \subparagraph{Consequence:} The misactor now has access to the user's log-in information and can from
now on spoof the user.


    \subparagraph{Reference to threat tree node(s):} ID\_df4, ID\_df7
    \subparagraph{Parent threat tree(s):} ID\_df, ID\_ds
    \subparagraph{DFD element(s):} TODO
    \subparagraph{Remarks:}
    \begin{enumerate}
        \item[r1.] TODO hmm, bij hun remarks staat er da dit mogelijk is wegens insecure data flow between entities en system.
						wij hebben hier wel SSL, dus is deze threat mogelijk?
        \item[r2.]
    \end{enumerate}



