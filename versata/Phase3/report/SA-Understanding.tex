\section{Understanding the architecture}
In this section a usage scenario in the provided ReMeS system will be explained by using the ReMeS components 
and client-server view diagrams.
The scenario in question is the scenario where an alarm arrives at the ReMeS back-end and the actuator is activated.

When an alarm trame arrives at the ReMeS back-end, first of all the incoming communication component will translate
	the stream of bits to a NativeDataTrame. Then the same component will analyze that NativeDataTrame to find out
	what kind of trame it is. This analyzation will conclude that it is indeed an alarm trame.

Since the trame is an alarm trame, the incoming communication component will call the method receiveAlarmTrame in the
	alarm processor. The alarm processor will store this alarm in the database and will receive the alarm configuration
	data from the database. If this data includes a customer notification, it will call the notifyAlarm method of the
	outgoing communication component. This outgoing communication component will then make sure that the customer gets
	notified about the alarm.

If the alarm configuration data includes a valve actuation, the actuator controller will be called by the alarm processor.
	The actuator controller will then create an actuator message for the specific valve and call the correct method in the
	outgoing communication component. If the incoming communication component receives an acknowledgement, this data will be
	stored in the database. If the incoming communication component doesn't receive an acknowledgement trame, the actuator controller
	will make the outgoing communication component resend the control trame. The actuator controller will attempt to do this
	a certain amount of times. If there is still no acknowledgement received, the notifyIssue method will be called in the
	outgoing communication component. The customer will be notified about the issue.

