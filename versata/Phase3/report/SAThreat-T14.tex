

\paragraph{T06 - Information disclosure internal process}
\label{par:t07}
% the attributes noted with an asterix (*) are obligated
    \subparagraph{Summary:} The misactor gains access to one of the internal processes.
    \subparagraph{Primary mis-actor:} Authorized insider
    \subparagraph{Basic path:}
    \begin{enumerate}
        \item[bf1.] The misactor has the required privileges to access to processes.
        \item[bf2.] The misactor uses his privileges to access information outside the scope of his job.
    \end{enumerate}
    \subparagraph{Consequence:} The misactor has access to (possibly sensitive) personal identifiable information.


    \subparagraph{Reference to threat tree node(s):} ID\_p
    \subparagraph{Parent threat tree(s):} ID\_p
    \subparagraph{DFD element(s):} All internal processes (7.2 - 7.13)
    \subparagraph{Remarks:}
    \begin{enumerate}
        \item[r1.] This threat especially applies to administrators of our system. Our databases have access control installed,
					but administrators will still have full access to all the data (assumption \ref{it:ass4}).
        \item[r2.] This threat is inspired by \textit{spoofing} an entity leaf threat, however, when an insider has to
much privileges, this threat applies as well. Spoofing entities with access to internal processes
is not considered, as we assume the system is physically protected(assumption \ref{it:ass1}).
		\item[r3.] We assume processes are not corruptable (assumption \ref{it:ass17}).
		\item[r4.] The side channel attack is described in \nameref{par:t08}.
    \end{enumerate}



