\section{Toepasbaarheid van MDA}
\subsection{Lessen}
Het moeten toepassen van een transformatie van Merode naar een oplossing binnen het versata framework, heeft geleid tot enkele
interessante inzichten. 
De voornaamste ervaring die als leerrijk te beschouwen is, is namelijk het feit dat op deze manier al geleerd kan worden hoe te werken met versata studio. 
Deze software heeft echter nog wel zijn mankementen, beiden in bruikbaarheid en in termen van intu\"itief gebruik. Hieruit valt echter wel af te leiden dat er toch een vrij hoge leercurve aanwezig is en dat er toch nog wel wat extra tijd nodig is om versata studio volledig onder de knie te krijgen.
Verder is het wel interessant geweest om eens kennis te maken met een manier om software op verschillende lagen op een relatief simpele manier te ontwikkelen. Een html web applicatie met een service backend in java, is gewoonlijk iets kostelijker om volledig zelf te ontwikkelen zonder hulp van dergelijke tools.
\subsection{Overige noodzakelijkheden}
Een aantal zaken zijn echter wel nog noodzakelijk voor het mechanisch transformeren van merode naar een oplossing binnen het versata framework. Deze zaken zijn voornamelijk contextgericht en worden daardoor niet echt ondersteund door merode.
Zo hebben we documenten nodig waarin de specifieke contextgerelateerde datavelden staan voor de business objecten. 
Maar ook context specifieke constraints waaraan bepaalde objecten moeten voldoen moeten op een of andere eenduidige manier worden aangeboden als input voor het transformatie process.
\subsection{Conclusies}
Zo is het wel duidelijk geworden dat het gebruik van MDA niet volledig nutteloos is voor het ontwikkelen van bepaalde toepassingen. 
Er is wel degelijk een bepaalde set van omstandigheden waarbinnen het nuttig is om te werken met een MDA aanpak. 
Hoewel het niet de meest leuke manier is van software schrijven, biedt MDA wel een manier aan om software te genereren zonder de technische onderlegging te hebben die meestal vereist is om dergelijke software zelf uit te werken. 
Op die manier is het ook voor mensen met een minder technische achtergrond om MDA toe te passen en werkende softwarepakketten te ontwikkelen.
